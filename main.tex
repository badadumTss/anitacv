\documentclass{resume}
\usepackage{fontawesome5}

\begin{document}

\fontfamily{qag}\selectfont

\noindent
\begin{tabularx}{\linewidth}{@{}m{0.8\textwidth} m{0.2\textwidth}@{}}
  {
    \Large{Anita Ghisandi} \newline
    \small{
      \clink{
        \href{mailto:anita.ghislandi@gmail.com}{anita.ghislandi@gmail.com}
        \textbf{$\cdot$}
        {\fontdimen2\font=0.75ex +39 339 103 0663}
      } \newline
    }
  } & 
  {
    \hfill
    \roundpic{3cm}{3cm}{images/pfp.jpg}
  }
\end{tabularx}
\begin{center}
  \begin{tabularx}{\linewidth}{@{}*{2}{X}@{}}
    % left side %
    {
      \csection{ESPERIENZE}{\footnotesize
        \begin{itemize}[label={}]

        \item \frcontent{Liceo Don Milani}{Progetto thelab} { Dialogo
            socratico con bambini delle elementari: aiutare i bambini
            a comprendere princpi base della scienza attarverso il
            dialogo } {2017 -- 2018}

        \item \frcontent{Sudafrica -- Johannesburg}{Intercultura} {
            Exchange student presso la townshiop di Soweto per un
            bimestre} {2016}
          
        \end{itemize}
      }
      \csection{FORMAZIONE}{\small
        \begin{itemize}[label={}]

        \item
          \frcontent{Maturità scientifica}
          {Liceo Don Lorenzo Milani, Romano di Lombardia}
          {
            Voto finale 95/100
          }
          {2013 -- 2018}
          
        \item
          \frcontent{Laurea triennale in Scenze psicologiche cognitive e psicobiologiche}
          {Università di Padova}
          {
            Voto finale 110/110 con lode
          }
          {2018 -- 2021}

        \item
          \frcontent{Laurea magistrale in Neuroscienze e \newline
            riabilitazione neuropsicologica}
          {Università di Padova}
          {}
          {2021 -- in corso}
        \end{itemize}
      }
      \csection{TESI TRINENNALE}{\small
        \begin{itemize}[label={}]
        \item \frcontent{Ci fidiamo anche degli sguardi sbagliati}
          {Università di Padova}
          {\footnotesize Uno studio
            sull'orientamento attentivo mediato da
            \textit{gaze-cueing}}
          {5 lug. 2021}
        \end{itemize}
        % \begin{itemize}
        %   \item \frcontent{Handling NFTs on ethereum and hotmoka}
        %     {University of Padova}
        %     {\footnotesize Bachelor thesis on developing a distributed
        %       application (\textit{dapp}) with ethereum and hotmoka as
        %       backend, and angular as frontend, to handle NFTs. The
        %       project resulted in a distributed marketplace, currently
        %       available via IPFS at
        %       \clink{marketplace.badadumTss.codes}}
        %     {30 sept. 2021}
        % \end{itemize}
      }
      % \csection{SKILLS}{\small
        % \begin{itemize}
        % \item \textbf{Patterns \& Practices} \newline
        %   {\footnotesize Imperative programming, object oriented
        %     programming, functional programming, CI \& CD,
        %     microservices, linux system administration}
        % \item \textbf{Languages} \newline
        %   {\footnotesize English: IELTS 7.5 grade, French: B2 with TCF}
        % \end{itemize}
      % }
    }
    % end left side %
    & 
    % right side %
    {
      \csection{CORSI AFFRONTATI}{\small
        \begin{itemize}[label={}]
        \item \textbf{Corso di canto} \newline {\footnotesize
            Accademia \textit{Lizard} fino alla licenza di secondo
            livello \newline 3 anni}
        \item \textbf{Corso di Teatro canto} \newline
          {\footnotesize \textit{I suoni dell'ascolto}
            \newline 1 anno -- 2020/21}
        \item \textbf{Corso base di Teatro} \newline
          {\footnotesize \textit{Animali da palco}
            \newline 1 anno -- 2021/22}
        \item \textbf{Corso di inglese} \newline
          {\footnotesize \textit{C2} Vacanza studio a Londra con EF
            \newline 2 settimane -- 2018}
        \item \textbf{Corso di francese} \newline
          {\footnotesize \textit{A2} Vacanza studio a Parigi con EF
            \newline 2 settimane -- 2017}
        \end{itemize}
      }
      \csection{HOBBY E INTERESSI}{\small
        \begin{itemize}[label={}]
        \item \textbf{Scrittura creativa} \newline
         {\footnotesize Corso online su domestika: \textit{scrittura
              creativa per principianti: dai vita alla tua storia}}
        \item \textbf{Canto} \newline {\footnotesize diverse
            esperienze come cantante in alcuni gruppi musicali}
        \end{itemize}
      }
      % \mbox{}
      % \vfill
      % \footnotesize {
      %   \hfill \href{https://github.com/badadumTss}{badadumTss} \faGithub
      % }
      % \mbox{}
      % \vfill
      % \footnotesize {
      %   \mbox{} \hfill \href{https://twitter.com/luca_zann}{luca\_zann} \faTwitter
      % }
    }
  \end{tabularx}
\end{center}
\end{document}